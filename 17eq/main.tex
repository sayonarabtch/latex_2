\documentclass[book,10pt]{book}
\usepackage[utf8]{inputenc}
\usepackage[english,russian]{babel}
\usepackage{indentfirst}
\usepackage[left=10mm, top=20mm, right=10mm, bottom=10mm, nohead, nofoot]{geometry}

\textwidth = 430pt

\begin{document}
\subsubsection{17 уравнений, которые изменили мир.}
\begin{tabular}[t]{lll}
 1. Теорема Пифагора & $a^{2}+b^{2}=c^{2}$ & Пифагор, 530 до н.э.  \\
 2. Логарифмы & $\log{xy}=\log{x} + \log{y} $ & Джон Непер, 1610 \\
 3. Приращение & $\frac{d f}{d t}=\lim\limits_{h\to 0} \frac{f(t+h)-f(t)}{h}$ & Ньютон, 1668 \\
 4. Закон Всемирного Тяготения & $F=G\frac{m_{1}m_{2}}{r^{2}}$ & Ньютон, 1667 \\
 5. Квадратный корень из минус единицы & $\iota^{2}=-1$ & Эйлер, 1750\\
 6. Формула Эйлера для многогранника & $V-E+F=2$ & Эйлер, 1751\\
 7. Нормальное распередление & $\Phi(x)=\frac{1}{\sqrt{2\pi\rho}}e^{\frac{(x-\mu)^2}{2\rho^{2}}}$ & К.Ф. Гаусс, 1810\\
 8. Волновое уравнение & $\frac{\partial^{2}u}{\partial{t}^{2}}=c^{2}\frac{\partial^{2}u}{\partial{x}^{2}}$ & Жан Лерон Д’Аламбер, 1746\\
 9. Преобразование Фурье & $f(\omega)=\frac{1}{\sqrt{2\pi}} \int\limits_{-\infty}^\infty f(x) e^{-ix\omega} d{x} $ & Жан Батист Жозеф Фурье, 1822\\
 10. Уравнение Навера-Стокса &  $\rho(\frac{\partial{\mathrm{v}}}{\partial{t}}+{\mathrm{v}}\cdot\nabla {\mathrm{v}})=-\nabla p+\nabla\cdot T+f$ & А. Навье, Д. Стокс, 1845\\
 11. Уравнения Максвелла & 
 \begin{tabular}[t]{ll} 
$\nabla \cdot  {\bfseries E}=\frac{\rho}{\epsilon_{0}}c$ & $\nabla \cdot {\bfseries H}=\frac{1}{c} \frac{\partial E}{\partial t}$ \\ 
$\nabla \cdot {\bfseries H}=\frac{1}{c} \frac{\partial E}{\partial t}$ & $\nabla \cdot  {\bfseries E}=\frac{\rho}{\epsilon_{0}}c$\\
 
  \end{tabular} & Д.К. Максвелл, 1865 \\\\
 12. Второй закон термодинамики & $d S \geq 0$ & Л. Больцман, 1874\\
 13. Теория относительности & $E=mc^{2} $ & Эйнштейн, 1905\\
 14. Уравнение Шрёдингера & $\iota \hbar \frac{\partial}{\partial t} \Psi = H \Psi$ & Э. Шрёдингер, 1927\\
 15. Теория Информации & $H = -\sum p(x) \log{p(x)}$ & К. Шеннон, 1949\\
 16. Теория хаоса &  $x_{t+1}=k x_{t} (1 - x_{t})$ &  Роберт Мэй, 1975\\
 17. Уравнение Блэка-Шоулза & $\frac{1}{2} \sigma^{2} S^{2} \frac{\partial^{2} \mathrm{V}}{\partial S^{2}}+r S \frac{\partial \mathrm{V}}{\partial S} + \frac{\partial \mathrm{V}}{\partial t} - r V = 0 $ &  Ф. Блэк, М.Шоулз, 1990 \\
\end{tabular}
\end{document} 